\documentclass[a4paper,11pt,notitlepage]{article}
\usepackage[utf8]{inputenc}	% latin2 - kodowanie iso-8859-2; cp1250 - kodowanie windows
\usepackage[T1]{fontenc}
\usepackage{enumitem}
\usepackage[polish]{babel}
\usepackage[MeX]{polski}
\selectlanguage{polish}

\usepackage{graphicx}

\hyphenation{FreeBSD}

\author{Wojciech Świstowski (252572)}
\title{Specyfikacja Funkcjonalna projektu Problem N-Ciał}
\date{\today}

\linespread{1.3}

\usepackage{indentfirst}

\begin{document}
\maketitle

\section{Cel projektu}

Celem projektu Problem N-Ciał jest stworzenie programu, który będzie obliczał pozycje ciał w trójwymiarowej przestrzeni dla kolejnych kroków czasowych.

\section{Funkcjonalności}
Program będzie realizował następujące funkcjonalności:
\begin{itemize}[noitemsep]
	\item Wczytywanie do programu jednego lub kliku plików z danymi zawierającymi parametry ciał (masę, położenie, prędkość)
	\item Ustawianie przy wywoływaniu programu parametrów symulacji: długości kroku czasowego, długości symulacji
	\item Obliczanie pozycji każdego wczytanego ciała w kolejnych krokach czasowych
	\item Generowanie zestawu plików, gdzie każdy kolejny plik zawiera położenie ciała w każdym kroku czasowym (ustawianego przez użytkownika)
	\item Pliki wygenerowane za pomocą programu będą mogły być użyte do wyświetlenia symulacji w programie Gnuplot
\end{itemize}

\section{Wywołanie}
Uruchomienie programu będzie się odbywało z linii poleceń według podanego przykładu:

\footnotesize\begin{verbatim}
% nBody 1000 3 h 2 dane1.txt dane2.txt ../wyniki/
\end{verbatim}\normalsize

Gdzie kolejne parametry wywołania programu to:
\begin{itemize}[noitemsep]
	\item Ilość iteracji po których nastąpi koniec symulacji. Musi być liczbą całkowitą mieszczącą się w przedziale 0 - 65 535 
	\item Długość kroku czasowego pomiędzy kolejnymi pozycjami ciał. Musi być liczbą całkowitą mieszczącą się w przedziale 0 - 65 535 (unsigned integer)
 	\item Jednostka czasu kroku czasowego. Dostepne są następujące opcje:
	\begin{itemize}[noitemsep]
		\item s - sekundy
		\item m - minuty
		\item h - godziny
		\item d - dni
		\item w - tygodnie
		\item y - lata
	\end{itemize}
	\item Liczba plików z danymi. 
	\item Ścieżka do pliku z danymi.
	\item Ścieżka do kolejnego pliku z danymi
	\item Parametr opcjonalny - ścieżka do katalogu z plikami wyjściowymi. Domyślnie jest to './results/'.
\end{itemize}

\section{Dane wejściowe}
Pliki służące jako dane wejściowe będą musiały mieć określony format, aby program mógł je przetworzyć. W przypadku niepoprawnego formatowania pliku, program poinformuje użytkownika o konieczności poprawienia błędnego pliku i wyświetli przykładowy plik.

\vspace{0.1in}
Obsługiwany format danych wejściowych:

\vspace{0.1in}

\footnotesize\begin{verbatim}
Cialo: "nazwa-ciała"
Masa: "masa-ciala"
Pozycja: "położenie-ciała-w-układzie-XYZ"
Predkosc: "wektor-predkosci-ciała-w-układzie-XYZ"
------
Cialo: "nazwa-ciała"
...
\end{verbatim}\normalsize
\vspace{0.1in}
Gdzie:
\begin{itemize}[noitemsep]
		\item "nazwa-ciala" może składać się wyłącznie z liter
		\item "masa-ciala" (podana w kilogramach) może być wyłącznie liczbą większą od 0  
		\item "położenie-ciała-w-układzie-XYZ" (podane w metrach) jest położeniem względem obserwowanego obiektu w chwili początkowej
		\item "wektor-predkosci-ciała-w-układzie-XYZ" (podane w metrach/sek) jest wektorem prędkości w chwili początkowej
		\item ------ jest obowiązkowym separatorem pomiędzy ciałami
	\end{itemize}

\vspace{0.1in}
Przykładowy plik z danymi zawierający 2 ciała:

\vspace{0.1in}

\footnotesize\begin{verbatim}
Nazwa: ziemia
Masa: 152134213
Pozycja: 5632134,516341421,235132421
Predkosc: 105123123,232154123,554214312
------
Nazwa: mars
Masa: 54217984371290
Pozycja: 3032421,2342154213,5251324123
Predkosc: 4031241231,214213213,53412121321
\end{verbatim}\normalsize
\section{Dane wyjściowe}
Danymi wyjściowymi będą pliki zawierające położenie ciał w przestrzeni. Dla każdego ciała stworzony zostanie oddzielny plik, który będzie zawierał kolejne położenia ciała. Pliki będą miały nazwy takie same jak obiekty dla których obliczane będzie położenie, a rozszerzeniem będzie '.txt'.

\vspace{0.1in}
Przykład pliku z danymi wyjściowymi:

\vspace{0.1in}
\begin{tabular}{llll}

\#& X & Y & Z\\
 & 24 & 45 & 23\\
 & 12 & -41 & -14\\
 & 34 & 26 & -7\\
 & -18 & 13 & 27\\
\end{tabular}

\section{Obsługa błędów}
Program będzie miał zaimplementowaną obsługę błędów w sposób pozwalający na szybkie ich zidentyfikowanie i poprawienie. W przypadku wystąpienia błędu użytkownik zostanie poinformowany o tym za pomocą odpowiedniego komunikatu.  

\vspace{0.1in}
Błędy, które będą obsłużone w programie:

\vspace{0.1in}
\begin{tabular}{p{3cm}|p{8cm}}

Przyczyna błędu& Sposób obsłużenia\\
\hline\hline
Niepoprawne wywołanie programu & Program wyświetli komunikat informujący który argument wywołał błąd. Następnie zakończy swoje działanie.\\
\hline
Niekompletne dane wejściowe & Program wyświetli komunikat informujący której danej brakuje. Następnie zakończy swoje działanie.\\
\hline
Niepoprawny typ danych wejściowych & Program wyświetli komunikat informujący które dane są niepoprawne, a następnie zakończy działanie.\\
\hline
Dane dublujące się & Program wyświetli komunikat które dane się dublują, a następnie zakończy działanie.\\
\hline
Niepoprawna ścieżka do folderu z wynikami & Program wyświetli komunikat o niepoprawnej ścieżce. Wyniki zostaną zapisane do folderu ./results/\\

\end{tabular}
\end{document}