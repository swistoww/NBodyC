\documentclass[a4paper,11pt,notitlepage]{article}
\usepackage[utf8]{inputenc}	% latin2 - kodowanie iso-8859-2; cp1250 - kodowanie windows
\usepackage[T1]{fontenc}
\usepackage{enumitem}
\usepackage[polish]{babel}
\usepackage[MeX]{polski}
\selectlanguage{polish}

\usepackage{graphicx}

\hyphenation{FreeBSD}

\author{Wojciech Świstowski (252572)}
\title{Specyfikacja Funkcjonalna}
\date{\today}

\linespread{1.3}

\usepackage{indentfirst}

\begin{document}
\maketitle

\section{Cel projektu}

Celem projektu jest zaimplementowanie programu, który będzie obliczał pozycje ciał w trójwymiarowej przestrzeni dla kolejnych kroków czasowych. 

\section{Funkcjonalności}
Program będzie realizował następujące funkcjonalności:
\begin{itemize}[noitemsep]
	\item Wyświetlenie sposobu obsługi programu
	\item Wczytywanie do programu jednego lub kliku plików z danymi
	\item Ustawianie przy wywoływaniu programu parametrów symulacji: długości kroku czasowego, długości symulacji
	\item Obliczanie pozycji każdego wczytanego ciała w kolejnych krokach czasowych
	\item Generowanie zestawu plików, gdzie każdy kolejny plik zawiera położenie planet po upływie określonego czasu (ustawianego przez użytkownika)
	\item Pliki wygenerowane za pomocą programu będą mogły być użyte do wyświetlenia symulacji w programie Gnuplot
	\item Program będzie generował podstawowy skrypt do symulacji w programie Gnuplot
\end{itemize}

\section{Wywołanie}
Uruchomienie programu będzie się odbywało z linii poleceń według podanego przykładu:

\footnotesize\begin{verbatim}
% nBody 3 h 1000 2 dane1.txt dane2.txt ../wyniki/
\end{verbatim}\normalsize

Gdzie kolejne parametry wywołania programu to:
\begin{itemize}[noitemsep]
	\item Długość kroku czasowego pomiędzy kolejnymi pozycjami ciał. Musi być liczbą całkowitą mieszczącą się w przedziale 0 - 65 535 (unsigned integer)\emph{(n1)}
 	\item Jednostka czasu kroku czasowego. Dostepne są następujące opcje:\emph{(nazwa pliku)}
	\begin{itemize}[noitemsep]
		\item s - sekundy
		\item m - minuty
		\item h - godziny
		\item d - dni
		\item w - tygodnie
		\item y - lata
	\end{itemize}
	\item Ilość iteracji po których nastąpi koniec symulacji. Musi być liczbą całkowitą mieszczącą się w przedziale 0 - 65 535 \emph{(n2)}
	\item Liczba plików z danymi. \emph{(n4)}
	\item Ścieżka do pliku z danymi.\emph{(n3)}
	\item Ścieżka do kolejnego pliku z danymi \emph{(n4)}
	\item Ścieżka do katalogu z plikami wyjściowymi. \emph{(n4)}
\end{itemize}

\section{Dane wejściowe}
Pliki służace jako dane wejściowe będą musiały mieć określony format, aby program mógł je przetworzyć. W przypadku niepoprawnego formatowania pliku, program poinformuje użytkownika o konieczności poprawienia błędnego pliku i wyświetli przykładowy plik.

\vspace{0.1in}
Obsługiwany format danych wejściowych:

\vspace{0.1in}

Name: <<nazwa_ciała>>\\
Mass: <masa_ciała>\\
Position: <<położenie_ciała_w_układzie_XYZ>>\\
Velocity: <<wektor_predkosci_ciała_w_układzie_XYZ>>\\
------\\
Name: <<nazwa_ciała>>\\
...\\

\vspace{0.1in}
Przykłądowy plik z danwi zawierający 2 ciała:

\vspace{0.1in}

Name: ziemia\\
Mass: 10\\
Position: 56,51,23\\
Velocity: 10,2,5\\
------\\
Name: mars\\
Mass: 5\\
Position: 30,23\\
Velocity: 40,21\\

\section{Dane wyjściowe}
Danymi wyjściowymi będą pliki zawierające położenie ciał w przestrzeni. Każdy plik będzie reprezentował kolejny upływ czasu (iterację) i zawierał położenia wszystkich ciał.

\vspace{0.1in}
Przykład pliku z danymi wyjściowymi dla 4 ciał:

\vspace{0.1in}
\begin{tabular}{llll}

\#& X & Y & Z\\
 & 24 & 45 & 23\\
 \\
 & 12 & -41 & -14\\
 \\
  & 34 & 26 & -7\\
 \\
  & -18 & 13 & 27\\
\end{tabular}

\section{Obsługa błedów}}
Program będzie miał zaimplementowaną obsługę błędów w sposób pozwalający na szybkie ich zidetyfikowanie i poprawienie. W przypadku wystąpienia błędu użytkownik zostanie poinformowany o tym za pomocą odpowiedniego komunikatu.  

\vspace{0.1in}
Błędy, które będą obsłużone w programie:

\vspace{0.1in}
\begin{tabular}{ll}

Przyczyna błędu& Sposób obsłużenia\\
\hline
Niepoprawne wywołanie programu & Program wyświetli komunikat informujący który argument wywołał błąd. Następnie zakończy swoje działanie.\\
Niekompletne dane wejściowe & Program wyświetli komunikat informujący której danej brakuje, a także wyświetli przykładowy plik z danymi wejściowymi. Następnie zakończy swoje działanie.\\
Niepoprawny typ danych wejściowych & Program wyświetli komunikat informujący które dane są niepoprawne, a następnie zakończy działanie.\\
Dane dublujące się & 2\\
Niepoprawnie sformatowany plik wejściowy & 2\\
Niepoprawna ścieżka do folderu z wynikami & 2\\
Niekompletne dane & 2\\
...\\
\end{tabular}


\section{Scenariusz działania}

\begin{enumerate}[noitemsep]
	\item Uruchomienie za pomocą linii poleceń.
	\begin{enumerate}
		\item W przypadku błędnych danych zostanie wyświetlony odpowiedni komunikat informujący o popełnionym błędzie
	\end{enumerate}
	\item W przypadku wczytania pliku zawierającego zwykły tekst program poprosi o podanie długości n-gramów
	\item Podzielenie tekstu na n-gramy
	\item Zapisanie n-gramów do pliku
	\item Wyliczenie informacji statystycznych na temat przetwarzanego tekstu
	\begin{enumerate}
		\item Jeżeli program wywołany był z parametrem \emph{ stat } zapisanie pliku statystycznego
	\end{enumerate}
	\item Wygenerowanie tekstu
	\item Zapisanie wygenerowanego tekstu do pliku
	\item Wyświetlenie informacji o zakończeniu symulacji
\end{enumerate}
\end{document}
